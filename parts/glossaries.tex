%%%%%%%%%%%%%%%%%%%%%%%%%%%%%%%%%%%%%%%%%%%%%%%%%%%%%%%%%%%%%%%%%%
% Glossary setup
%%%%%%%%%%%%%%%%%%%%%%%%%%%%%%%%%%%%%%%%%%%%%%%%%%%%%%%%%%%%%%%%%%

% \setglossarystyle{mcolindexgroup} % only with \usepackage{glossary-mcols}
\setglossarystyle{indexgroup}

% Make glossary
\makeglossaries

% Acronym entries
\newacronym{1d}{1D}{1-dimensional}
\newacronym{2d}{2D}{2-dimensional}
\newacronym{3d}{3D}{3-dimensional}
\newacronym{tpc}{TPC}{time projection chamber}
\newacronym{lar}{LAr}{liquid argon}
\newacronym{gar}{GAr}{gaseous argon}
\newacronym{lartpc}{LArTPC}{liquid argon time projection chamber}
\newacronym{pmt}{PMT}{photomultiplier tube}
\newacronym{hv}{HV}{high voltage}
\newacronym{let}{LET}{linear energy transfer}
\newacronym{bp}{BP}{boiling point}
\newacronym{mp}{MP}{melting point}
\newacronym{tp}{TP}{triple point}
\newacronym{cp}{CP}{critical point}
\newacronym{mip}{MIP}{minimally ionising particle}
\newacronym{bnb}{BNB}{Booster Neutrino Beam}
\newacronym{fnal}{FNAL}{Fermi National Accelerator Laboratory or Fermilab}
\newacronym{linac}{Linac}{linear accelerator}
\newacronym{sm}{SM}{Standard Model of particle physics}
\newacronym{cl}{CL}{confidence level}
\newacronym{rms}{RMS}{root mean square}
\newacronym{lhep}{LHEP}{Laboratory for High Energy Physics}
\newacronym{dc}{DC}{direct current}
\newacronym{ac}{AC}{alternating current}
\newacronym{petc}{PET-C}{polyethylene terephthalate, partially crystalline}
\newacronym{tpb}{TPB}{tetraphenyl butadien}
\newacronym{sipm}{SiPM}{silicon photomultiplier}
\newacronym{uv}{UV}{ultra violet}
\newacronym{mc}{MC}{monte carlo}
\newacronym{crt}{CRT}{cosmic-ray tagger}
\newacronym{cry}{CRY}{cosmic-ray shower generator}
\newacronym{geant}{Geant4}{\textbf{ge}ometry \textbf{an}d \textbf{t}racking 4}
\newacronym{gdml}{GDML}{geometry description markup language}
\newacronym{genie}{GENIE}{\textbf{G}enerates \textbf{E}vents and for \textbf{N}eutrino \textbf{I}nteraction \textbf{E}xperiments}
\newacronym{corsika}{CORSIKA}{\textbf{co}smic \textbf{r}ay \textbf{si}mulations for \textbf{ka}scade}
\newacronym{art}{ART}{\textbf{a}nalysis \textbf{r}econstruction \textbf{t}ools}
\newacronym{asl}{ASL}{above sea level}
\newacronym{lee}{LEE}{low energy excess}
\newacronym{pot}{POT}{protons on target}
\newacronym{slac}{SLAC}{Stanford Linear Accelerator Center}
\newacronym{pep}{PEP}{positron-electron-proton}
\newacronym{mwpc}{MWPC}{multi-wire proportional chamber}
\newacronym{em}{EM}{electronmagnetic}
\newacronym{sbn}{SBN}{short-baseline neutrino program}
\newacronym{rfq}{RFQ}{radio-frequency quadrupole}
\newacronym{rf}{RF}{radio-frequency}
\newacronym{lapd}{LAPD}{Liquid Argon Purity Demonstrator}
\newacronym{pcb}{PCB}{printed circuit board}
\newacronym{asic}{ASIC}{application-specific integrated circuit}
\newacronym{fpga}{FPGA}{field-programmable gate array}
\newacronym{daq}{DAQ}{data acquisition}
\newacronym{adc}{ADC}{analogue-to-digital converter}
\newacronym{fem}{FEM}{front-end readout module}
\newacronym{sram}{SRAM}{static random-access memory}
\newacronym{wls}{WLS}{wavelength shifting} % TODO Check if accronym can be used somewhere else not only for CRT
\newacronym{feb}{FEB}{front-end electronics board}
\newacronym{cpu}{CPU}{central processing unit}
\newacronym{lcs}{LCS}{laser calibration system}
\newacronym{ndyag}{Nd:YAG}{neodymium-doped yttrium aluminium garnet}
\newacronym{pai}{PAI}{Poliamide-imide}
\newacronym{cf}{CF}{ConFlat}
\newacronym{lartf}{LArTF}{liquid argon test facility}
\newacronym{sbnd}{SBND}{short-baseline near detector}
\newacronym{lsnd}{LSND}{liquid scintillator neutrino detector}
\newacronym{cc}{CC}{charged current}
\newacronym{nc}{NC}{neutral current}
\newacronym{qe}{QE}{quasi-elastic}
\newacronym{ccqe}{CCQE}{charge current quasi-elastic}
\newacronym{res}{RES}{resonant production}
\newacronym{dis}{DIS}{deep inelastic scattering}
\newacronym{fv}{FV}{fiducial volume}
\newacronym{pe}{PE}{\textbf{p}hoto\textbf{e}lectron}
\newacronym{roi}{ROI}{regions of interest}
\newacronym{fom}{FOM}{figure of merit}
\newacronym{mec}{MEC}{meson exchange current}
\newacronym{esf}{ESF}{effective spectral function}
\newacronym{tem}{TEM}{transverse enhancement model}
\newacronym{fsi}{FSI}{final-state interactions}
\newacronym{rfg}{RFG}{relativistic Fermi gas}
\newacronym{cms}{CMS}{centre of mass system}
\newacronym{svm}{SVM}{suport vector machine}
\newacronym{csda}{CSDA}{continuously slowing down approximation}
\newacronym{mcs}{MCS}{multiple Coulomb scattering}
\newacronym{pid}{PID}{particle identification}
\newacronym{CP}{CP}{charge conjugation parity}


\newacronym{neut}{NEUT}{Neutrino interaction simulation library, a neutrino generator}
\newacronym{nuwro}{NuWro}{Wroclaw Neutrino Event Generator, a neutrino generator}
\newacronym{gibuu}{GiBUU}{The Giessen BUU Project, a neutrino generator}
\newacronym{t2k}{T2K}{Tokai To Kamioka}
\newacronym{sno}{SNO}{Sudbury Neutrino Observatory}
\newacronym{dune}{DUNE}{Deep Underground Neutrino Experiment}

% Glossary entries
\newglossaryentry{Meson}{name={meson},description={Particle consisting of a quark antiquark pair $q\bar{q}^\prime$ connected by a gluon. A famous example are the charged pions: $\pi^+ = (u\bar{d})$ and $\pi^- = (d\bar{u})$}}
\newglossaryentry{Boson}{name={boson},description={Particles with integer spin which follow the Bose-Einstein statistics}}
\newglossaryentry{Fermion}{name={fermion},description={Particles featuring half-integer spin. They follow the Fermi-Dirac statistics}}
\newglossaryentry{quasifreeelectron}{name={quasi-free electron},description={Electrons that move with minimal interference through matter, \ie they almost move freely. The term suggests that free electrons only exist in vacuum \cite{NobleGasDetectors}}}
\newglossaryentry{excimer}{name={excimer},description={An electronically excited dimer, non-bonding in the ground state. In other words, a complex formed by the interaction of an excited molecular entity with a ground state partner of the same structure, \eg \ch{Ar2^*} \cite{GoldBook}}}
\newglossaryentry{singletexcimer}{name={singlet excimer},description={An excimer in a singlet energy state, \eg \ce{\ch{Ar2^*}(^1\Sigma^2_u)}},see={excimer}}
\newglossaryentry{tripletexcimer}{name={triplet excimer},description={An excimer in a triplet energy state, \eg \ce{\ch{Ar2^*}(^3\Sigma^2_u)}},see={excimer,singletexcimer}}
\newglossaryentry{selfabsorption}{name={self-absorption},description={The absorption of a photon in the same medium as it was emitted, \eg \ce{\ch{Ar^*} <=> Ar + $\gamma_{\text{res}}$}. The absorbed energy is usually dissipated by collisional transfer of energy, or through emission of photons of the same or other frequencies. In consequence, the observed radiant intensity of a spectral line (or band component) emitted by a source may be less than the radiant intensity would be from an optically thin source having the same number of emitting atoms \cite{GoldBook}}}
\newglossaryentry{selftrapping}{name={self-trapping},description={A process of immobilisation of an ion by its own lattice distortion field in a crystal. In a rare gas  \ce{R} this leads to the formation of charged dimers, \ie \ce{\ch{R^+} + R -> \ch{R2^+} + R} \cite{Self-Trapping}.}}
\newglossaryentry{dynamicaltrapping}{name={dynamical trapping},description={A resonant transfer relaxation process in which two neighbouring atoms lose sufficient energy to become stabilised in a lower virbational state of an excimer \cite{LArSelf-Trapping}. The process is similar to self-trapping. For a substance \ce{R} dynamical trapping can be written as \ce{\ch{R^*} + 2R -> \ch{R2^*} + R}},see={excimer,selftrapping}}
\newglossaryentry{electronegativity}{name={electronegativity},plural={electronegativities},description={The power of an atom to attract electrons to itself. It is dimensionless entity and denoted by $\chi_\text{r}$. The scale is chosen so as to make the relative electronegativity of hydrogen $\chi_\text{r}(\ce{H}) = 2.1$ \cite{GoldBook}}}
\newglossaryentry{mobility}{name={mobility},plural={mobilities},description={Denoted by the symbol $\mu$ mobility indicates how mobile a charge carrier is in a given medium, \eg in \gls{lar}. Mobility is temperature $T$ and electric field strength $E$ dependent. It is defined by the drift velocity of a charge carrier divided by the electric field strength $\mu \coloneqq v(E)/E$ \cite{GoldBook}}}
\newglossaryentry{zero-fieldmobility}{name={zero-field mobility},description={It denotes the mobility at low electric field strengths, \ie $E \to 0$. Zero-field mobility has the symbol $\mu_0$ and is constant in $E$ although still temperature $T$ dependent. It is defined as $\mu_0 \coloneqq \lim \limits_{E \to 0} v(E)/E$ \cite{NobleGasDetectors}}, see={mobility}}
\newglossaryentry{InductionPlane}{name={induction plane}, description={A readout wire plane which is transparent to drift electrons. The latter then induce a bipolar current signal while passing a wire of said plain}}
\newglossaryentry{CollectionPlane}{name={collection plane}, description={A readout wire plane which is not transparent to drift electrons. The latter are collected by the wires of said plain leading to an unipolar signal}}
\newglossaryentry{DielectricStrength}{name={dielectric strength}, description={The maximum electric field a material is able to withstand an electrical breakdown and thus becoming conductive}}
\newglossaryentry{WorkFunction}{name={work function}, description={The minimum work needed to extract electrons from the Fermi level of a metal across a surface carrying no net charge \cite{GoldBook}}}
\newglossaryentry{VanDerWaalsForce}{name={van der Waals Force}, description={The attractive or repulsive force between molecular entities (or between groups within the same molecular entity) other than those due to bond formation or to the electrostatic interaction of ions or of ionic groups with one another or with neutral molecules. The term includes: dipole–dipole, dipole-induced dipole and London (instantaneous induced dipole-induced dipole) forces. The term is sometimes used loosely for the totality of nonspecific attractive or repulsive intermolecular forces \cite{GoldBook}}}
\newglossaryentry{G10}{name={G-10}, description={A glass-based epoxy resin laminate often used in circuit boards}}
\newglossaryentry{Photolithography}{name={photolithography}, description={A technique that uses light to produce nanometre scale electronic circuits on silicon wavers, \ie so-called integrated circuits}}
\newglossaryentry{RingBuffer}{name={ring buffer}, description={also circular buffer, refers to an fixed size area in memory with an address space connected end-to-end. When the buffer is filled, new data is written starting at the beginning of the buffer and overwriting the old}} %TODO Text partially from boost.org
\newglossaryentry{Waveplate}{name={waveplate}, description={A device to alter the polarisation of of light. There are two common types: \textlambda/2-waveplates which change the light wave's linear polarisation, and \textlambda/4-waveplates which convert linearly polarised light into circularly polarised light and vice versa},see={LinearPolarisation}}
\newglossaryentry{PockelsCell}{name={Pockels cell}, description={Consists of a nonlinear crystal in which a \gls{dc} voltage introduces a change in refracting index. This can be used to create a switchable \textlambda/4-waveplates \cite{LaserTheory}},see={Waveplate}}
\newglossaryentry{GaussianMirror}{name={Gaussian mirror}, description={This is a so-called variable reflectivity mirror. It features a Gaussian reflectivity profile along its radius \cite{LaserTheory}}}
\newglossaryentry{DichroicMirror}{name={dichroic mirror}, description={Mirrors with significantly different reflection or transmission properties at two different wavelengths \cite{LaserEncyclopedia}}}
\newglossaryentry{BrewsterAngle}{name={Brewster angle}, description={An angle of incidence at which there is no reflection of p-polarised light at an uncoated optical surface \cite{LaserEncyclopedia}},see={LinearPolarisation}}
\newglossaryentry{Discriminator}{name={discriminator}, description={An electrical device which compares the strength of an input signal to a set threshold. The output is a single digital bit, discriminating if the threshold was exceeded or not, \ie the input signal is binarised.}}
\newglossaryentry{LinearPolarisation}{name={linear polarisation}, description={Linear direction of the electric field oscillation of a light beam. The linearly polarisation beam has two extreme states. In a s-polarised state, the electric field oscillates perpendicular to a reference plane, while it oscillates parallel to a reference plane in a p-polarised state \cite{LaserEncyclopedia}}}
\newglossaryentry{BeamGate}{name={beam gate}, description={An electronic binary signal, produced by a beam monitor. It features the width of a beam pulse and is used as a condition for triggering beam related events. In MicroBooNE the beam gate has a length of \SI{1.6}{\micro\meter} \cite{BNBBeamFlux}}}
\newglossaryentry{OpticalHit}{name={optical hit}, description={A scintillation light signal recorded by a single \gls{pmt}}}
\newglossaryentry{Flash}{name={flash}, plural={flashes}, description={Multiple correlated optical hits, produced by a single particle event in a \gls{lartpc}}, see={OpticalHit}}
\newglossaryentry{LArSoft}{name={LArSoft}, description={A toolkit to perform simulation, analysis and reconstruction with \glspl{lartpc}, developed at \gls{fnal} \cite{LArSoftWeb,LArSoft}}}
\newglossaryentry{Pandora}{name={Pandora}, description={A software development kit for pattern recognition. It is used to identify and characterise particle tracks, \ia, in a \gls{lartpc} \cite{Pandora}}}
\newglossaryentry{Vertex}{name={vertex}, plural={vertices}, description={A point in time and space at which a particle interaction occurs}}
\newglossaryentry{Cluster}{name={cluster}, description={A set of reconstructed particle detector signals which are expected to correlate to each other in time and space. Note, that a cluster is solely a reconstruction data product and therefore subject to misidentification. In MicroBooNE, clusters are a collection of correlated \gls{2d} wire hits}}
\newglossaryentry{Spline}{name={spline}, description={The result of an interpolation method which uses piecewise polynomial functions. The degree of the spline, is also the degree of the polynomials used.}}
% TODO Gaussian beam
% TODO Tunnel ionisation
% TODO Elaborate on the late discovery of the tripet and its prohibited decay path -> connection to singlet
% TODO Mabye define W-Value
% TODO Quasi-free electrons
% TODO Active volume
% TODO Ionisation
% TODO photoelectric effect
% TODO vertical dept!
% TODO Q-Switch?
% TODO Polariser
% TODO Delta ray
% TODO separate bosons and gauge bosons!!!
