%%%%%%%%%%%%%%%%%%%%%%%%%%%%%%%%%%%%%%%%%%%%%%%%%%%%%%%%%%%%%%%%%%
% STYLE AND LAYOUT
%%%%%%%%%%%%%%%%%%%%%%%%%%%%%%%%%%%%%%%%%%%%%%%%%%%%%%%%%%%%%%%%%%

% Page geometry
\usepackage[
%	papersize={6.5in,9in},
	hmargin={1.5in,1.5in},
	vmargin={1.2in,1in},
	ignoreheadfoot
]{geometry}

% Style
% \usepackage{palatino} % Use a different font (still a classical serif)
% \usepackage[parfill]{parskip} % Enter skips after paragraphs instead of indendation
% \raggedbottom % all pages have the height of the text on that page

% Define nice headers with a ruled line
\usepackage{fancyhdr}
\pagestyle{fancy}
\fancyhead{} %clear default fancy header style
%\fancyfoot{} %clear default fancy footer style
\fancyhead[RO]{\slshape \thepage} % odd pages, right
\fancyhead[LO]{\slshape \rightmark} % odd pages, left
\fancyhead[LE]{\slshape \thepage} % even pages, right
\fancyhead[RE]{\slshape \leftmark} % even pages, left
\renewcommand{\headrulewidth}{0.4pt}
%\renewcommand{\footrulewidth}{0.4pt}


%%%%%%%%%%%%%%%%%%%%%%%%%%%%%%%%%%%%%%%%%%%%%%%%%%%%%%%%%%%%%%%%%%
% PACKAGE IMPORTS
%%%%%%%%%%%%%%%%%%%%%%%%%%%%%%%%%%%%%%%%%%%%%%%%%%%%%%%%%%%%%%%%%%
\usepackage{setspace} % Control line spacing
\usepackage{hyperref}
	 \hypersetup{
	 colorlinks,
	 breaklinks=true,
	 linkcolor=blue,
	 anchorcolor=blue,
	 citecolor=blue,
	 urlcolor=blue,
} % enable tags and links and references
\usepackage[british]{babel} % English motherfucker, do you speak it?
\usepackage{csquotes} % Removes warning when babel is used with biblatex
\usepackage[usenames,dvipsnames]{xcolor} % allows colour definitions
\usepackage{graphicx} % place images
% \usepackage{svg} % include svg graphics format
\usepackage{rotating} % allows for figures/tables to be placed sideways
\usepackage[small,bf]{caption} % more verbose captions, use small/bold font
\usepackage{tocloft}[titles] % better table of contents
\usepackage{xspace} % for dynamic spaces (e.g. none before a colon, ...)
\usepackage{listings} % listings environment for source code formatting

% Use Biber as Biblatex backend
\usepackage[backend=biber,
            natbib=true,
%             style=phys, % Physics style
%             articletitle=false,biblabel=brackets,% APS style if phys
%             chaptertitle=false,pageranges=false,% APS style if phys
%             style=nature,
%             style=ieee,
            style=numeric,
            sorting=none]
            {biblatex}

\usepackage{tocbibind} % adds Bib, LoF, LoT, LoL to the ToC
\usepackage{expl3} % code for lookahead functionality of latin abbrev.
\usepackage{ifthen} % for definitions of edit and comment macros
\usepackage{amssymb} % some more symbols are available
\usepackage{amsmath} % math package
\usepackage{amsxtra} % math package
\usepackage{mathtools} % math package
\usepackage[low-sup]{subdepth} % correct for subscript shift when having a superscript
\usepackage{mathbbol} % Fancy math font (Identity matrix)
\usepackage[mathscr,scaled=1.15]{urwchancal} % Fancy math font, for lowercase operatiors 
\usepackage{physics} % Why? Because I'm a fucking Physicist, that's why!!!
\usepackage{braket} % use bra-ket vectors
% \usepackage{tensor} % Field theory tensor syntax
\usepackage{slashed} % Feynman slash operators = \gamma_{\mu}A^{\mu}
\usepackage{leftidx} % Left sub- and superscripts. Nicer (symmetrical) type set for subscripts in formulas
\usepackage{siunitx} % SI units
\usepackage[version=4]{mhchem} % chemistry symbols
\usepackage{chemformula} % other chemistry symbols package for nicer excited states, but everything els is rubbish
\usepackage{textgreek} % greek text symbols
\usepackage{textcomp} % more symbols in text form e.g. \textonehalf 
% \usepackage{xfrac} % same as textcomp but nicer? e.g. \textonehalf = \sfrac{1}{2}
\usepackage{simpler-wick} % Contractions
\usepackage{simplewick}
\usepackage{feynmp} % Feynman diagrams
\usepackage{subfig} % Subfigures
% \usepackage{subcaption} % Subfigure and Subcaption
\usepackage{pdfpages} % Insert whole PDF pages
\usepackage[normalem]{ulem} % for \sout
\usepackage{tabu} % Better tables (more control)
\usepackage{tabularx} % Best tables (more control)
\usepackage{multirow} % Multirow table entries
\usepackage{booktabs} % More table options
\usepackage[toc,acronym]{glossaries} % Glossary (create table of content entry and use acronym index)
\usepackage[xindy]{glossaries-extra} % Use Glossary extras for resetting at chapters
\usepackage{datetime2} % Dates
\usepackage{bm} % Bold mathstyle symbols, \hat for unit vectors
\usepackage{lineno} % Line numbers for proofreading
\usepackage[strict]{changepage} % For adjustwidth with subfloats
\usepackage{outlines} % For better staged itemize
\usepackage[
type={CC},
modifier={by-nc-sa},
version={4.0},
]{doclicense} % Creative commons license
% \usepackage{lmodern} % For quotes
% \usepackage{hyperref}
% % \usepackage[pagebackref=true]{hyperref}
% 	 \hypersetup{
% 	 colorlinks,
% 	 breaklinks=true,
% 	 linkcolor=blue,
% 	 anchorcolor=blue,
% 	 citecolor=blue,
% 	 urlcolor=blue,
% } % enable tags and links and references

%%%%%%%%%%%%%%%%%%%%%%%%%%%%%%%%%%%%%%%%%%%%%%%%%%%%%%%%%%%%%%%%%%
% Create additional math symbols
%%%%%%%%%%%%%%%%%%%%%%%%%%%%%%%%%%%%%%%%%%%%%%%%%%%%%%%%%%%%%%%%%%
% Create \mathpzc font, which is a lowercase caligraphy font used for operators
\DeclareFontFamily{OT1}{pzc}{}
\DeclareFontShape{OT1}{pzc}{m}{it}
                 {<-> s * [1.15] pzcmi7t}{}
\DeclareMathAlphabet{\mathpzc}{OT1}{pzc}{m}{it}

% Create sign function
\DeclareMathOperator{\sgn}{sgn}

%Math abbreviationis 
\newcommand{\Lagr}{\mathcal{L}}
\newcommand{\Hami}{\mathcal{H}}

% C++ signe without linebreak
\def\cpp{{C\nolinebreak[4]\hspace{-.05em}\raisebox{.4ex}{\tiny\bf ++}}\xspace}

%Math conventions
% \let\vec\mathbf % vectors are denoted as boldface characters and do not have a arrow on top (this is viewd as standard for typeset version by many (arrow only in handwriting))

%%%%%%%%%%%%%%%%%%%%%%%%%%%%%%%%%%%%%%%%%%%%%%%%%%%%%%%%%%%%%%%%%%
% Change Setups
%%%%%%%%%%%%%%%%%%%%%%%%%%%%%%%%%%%%%%%%%%%%%%%%%%%%%%%%%%%%%%%%%%
%\sisetup{detect-all}
\sisetup{separate-uncertainty}
% Date format
\DTMusemodule{british}{en-GB}


%%%%%%%%%%%%%%%%%%%%%%%%%%%%%%%%%%%%%%%%%%%%%%%%%%%%%%%%%%%%%%%%%%
% VARIABLE DEFINITIONS
%%%%%%%%%%%%%%%%%%%%%%%%%%%%%%%%%%%%%%%%%%%%%%%%%%%%%%%%%%%%%%%%%%

% =============
% Use big initial characters for autoref tags
\def\chapterautorefname{Chapter}
\def\sectionautorefname{Section}
\def\subsectionautorefname{Subsection}
\def\subsubsectionname{Subsubsection}

% =============
% 'Variables' for figure placement
\newcommand{\graphwidth}{0.9\textwidth}
\newcommand{\figurewidth}{0.9\linewidth}
\newcommand{\tablewidth}{1.2\linewidth}
\newcommand{\spacebeforegraph}{-5pt}
\newcommand{\spacebeforegraphcaption}{-10pt}

% =============
% Variables to toggle comments and edits
\newboolean{showedits}
\setboolean{showedits}{true} % toggle to show or hide edits
\newboolean{showcomments}
\setboolean{showcomments}{true} % toggle to show or hide comments

%%%%%%%%%%%%%%%%%%%%%%%%%%%%%%%%%%%%%%%%%%%%%%%%%%%%%%%%%%%%%%%%%%
% SQUARE BOX MACRO
%%%%%%%%%%%%%%%%%%%%%%%%%%%%%%%%%%%%%%%%%%%%%%%%%%%%%%%%%%%%%%%%%%

% Gray text box macro
% arg1: title, arg2: content
\newcommand{\squarebox}[2]{
\vspace{5pt}
\noindent\fcolorbox{white}{black!10}{%
    \minipage[t]{\dimexpr\linewidth-2\fboxsep-2\fboxrule\relax}
\textbf{#1} --- #2
    \endminipage}%
\vspace{1pt}
}

%%%%%%%%%%%%%%%%%%%%%%%%%%%%%%%%%%%%%%%%%%%%%%%%%%%%%%%%%%%%%%%%%%
% Macros for Quotation and Example-Story formatting
%%%%%%%%%%%%%%%%%%%%%%%%%%%%%%%%%%%%%%%%%%%%%%%%%%%%%%%%%%%%%%%%%%

% Environment for formatting quotes
% arg1: quote, arg2: source of the quote
\newcommand{\quoting}[2]{
	\begin{quote}
		``#1'' #2
	\end{quote}
}

% Environment for formatting of example stories
\newenvironment{example}{
	%\small
	\noindent
	%\mbox{}
	\slshape
	\quote
}{
	\vspace{\baselineskip}
}

%%%%%%%%%%%%%%%%%%%%%%%%%%%%%%%%%%%%%%%%%%%%%%%%%%%%%%%%%%%%%%%%%%
% MARKUP MACROS: PROOF READING
%%%%%%%%%%%%%%%%%%%%%%%%%%%%%%%%%%%%%%%%%%%%%%%%%%%%%%%%%%%%%%%%%%

\ifthenelse{\boolean{showedits}}
{
	\newcommand{\ugh}[1]{\textcolor{red}{\uwave{#1}}}  % please rephrase
	\newcommand{\ins}[1]{\textcolor{blue}{\uline{#1}}} % please insert
	\newcommand{\del}[1]{\textcolor{red}{\sout{#1}}}   % please delete
	\newcommand{\chg}[2]{\textcolor{red}{\sout{#1}}{$\rightarrow$}\textcolor{blue}{\uline{#2}}} % please change
}{
	\newcommand{\ugh}[1]{#1} % please rephrase
	\newcommand{\ins}[1]{#1} % please insert
	\newcommand{\del}[1]{}   % please delete
	\newcommand{\chg}[2]{#2} % please change
}

%%%%%%%%%%%%%%%%%%%%%%%%%%%%%%%%%%%%%%%%%%%%%%%%%%%%%%%%%%%%%%%%%%
% MARKUP MACROS: COMMENTS
%%%%%%%%%%%%%%%%%%%%%%%%%%%%%%%%%%%%%%%%%%%%%%%%%%%%%%%%%%%%%%%%%%

\newcommand{\id}[1]{$-$ \textsc{\rversion} $-$}
\newcommand{\yellowbox}[1]{\noindent\fcolorbox{gray}{yellow}{\bfseries\sffamily\scriptsize#1}}
\newcommand{\triangles}[1]{{\sf\small$\blacktriangleright$\textit{#1}$\blacktriangleleft$}}
\ifthenelse{\boolean{showcomments}}
{ 
	\newcommand{\ob}[2]{ % an orange box with following text
		{\noindent\colorbox{RedOrange}
			{\bfseries\sffamily\scriptsize\textcolor{white}{#1}}}
		{\textcolor{RedOrange}
			{\sf\small$\blacktriangleright$\textit{#2}$\blacktriangleleft$}}
	}
	\newcommand{\bb}[2]{ % an yellow box with following text
		{\noindent\colorbox{MidnightBlue}
			{\bfseries\sffamily\scriptsize\textcolor{white}{#1}}}
		{\textcolor{MidnightBlue}
			{\sf\small$\blacktriangleright$\textit{#2}$\blacktriangleleft$}}
	}
	\newcommand{\rb}[2]{% an red box with following text
		{\noindent\colorbox{BrickRed}
			{\bfseries\sffamily\scriptsize\textcolor{white}{#1}}}
		{\textcolor{BrickRed}
			{\sf\small$\blacktriangleright$\textit{#2}$\blacktriangleleft$}}
	}
	\newcommand{\gb}[2]{ % an green box with following text
		{\noindent\colorbox{Green}
			{\bfseries\sffamily\scriptsize\textcolor{white}{#1}}}
		{\textcolor{Green}
			{\sf\small$\blacktriangleright$\textit{#2}$\blacktriangleleft$}}
	}
	\newcommand{\version}{\noindent\emph{\scriptsize\id}}
}
{
	\newcommand{\ob}[2]{}
	\newcommand{\rb}[2]{}
	\newcommand{\bb}[2]{}
	\newcommand{\gb}[2]{}
	\newcommand{\version}{}
}

% =============
% Macros for working with this document...
\newcommand{\here}{\yellowbox{$\Rightarrow$ CONTINUE HERE $\Leftarrow$}}
\newcommand\fix[1]{\rb{FIX}{#1}} % fix this part
\newcommand\todo[1]{\ob{TO DO}{#1}} % things to do
\newcommand\think[1]{\bb{THINK}{#1}} % think about this
\newcommand\revcom[1]{\gb{@Reviewer}{#1}} % comment to the reviewers
\newcommand{\findQ}[1]{\ob{FIND QUOTE}{#1}} % find a quote for this 

%%%%%%%%%%%%%%%%%%%%%%%%%%%%%%%%%%%%%%%%%%%%%%%%%%%%%%%%%%%%%%%%%%
% LATIN ABBREVIATION MACROS
%%%%%%%%%%%%%%%%%%%%%%%%%%%%%%%%%%%%%%%%%%%%%%%%%%%%%%%%%%%%%%%%%%

% This handles punctuation and spacing
\ExplSyntaxOn
\newcommand{\latinabbrev}[1]{
  \peek_meaning:NTF .
  {#1\xspace}
  {
  	#1.\xspace
  }
}
% Same as above but with additional \emph{} style
\newcommand{\latinabbrevstyled}[1]{
  \peek_meaning:NTF .
  {\emph{#1}\xspace}
  {
  	\emph{#1.}\xspace
  }
}
\ExplSyntaxOff

% Use these macros in text
\newcommand{\ie}{\latinabbrevstyled{i.e}}
\newcommand{\eg}{\latinabbrevstyled{e.g}}
\newcommand{\etc}{\latinabbrevstyled{etc}}
\newcommand{\ia}{\latinabbrevstyled{i.a}}
\newcommand{\cf}{\latinabbrevstyled{cf}}
\newcommand{\etal}{\latinabbrevstyled{et\ al}}
\newcommand{\perse}{\latinabbrevstyled{per\ se}}


% \DeclareGraphicsExtensions{.pdf,.png,.jpg}

%%%%%%%%%%%%%%%%%%%%%%%%%%%%%%%%%%%%%%%%%%%%%%%%%%%%%%%%%%%%%%%%%%
% Additional SI units
%%%%%%%%%%%%%%%%%%%%%%%%%%%%%%%%%%%%%%%%%%%%%%%%%%%%%%%%%%%%%%%%%%

\DeclareSIUnit\ppm{ppm}
\DeclareSIUnit\ppb{ppb}
\DeclareSIUnit\ppt{ppt}
\DeclareSIUnit\erg{erg}
\DeclareSIUnit\atmosphere{atm}
\DeclareSIUnit\year{a}
\DeclareSIUnit\lightspeed{\textit{c}}
\DeclareSIUnit\fps{fps}
\DeclareSIUnit\pot{POT}

% Equation numbering level
\numberwithin{equation}{section}

%%%%%%%%%%%%%%%%%%%%%%%%%%%%%%%%%%%%%%%%%%%%%%%%%%%%%%%%%%%%%%%%%%
% Feynman diagram settings
%%%%%%%%%%%%%%%%%%%%%%%%%%%%%%%%%%%%%%%%%%%%%%%%%%%%%%%%%%%%%%%%%%

% In order to be able to read mp files for Feynman diagrams
\DeclareGraphicsRule{*}{mps}{*}{}

% Introduce a small momentum arrow with 5 variables:
% 1 Identifier/name, 2 position of the arrow (up, down, left, right), 3 label position (top, bot, lft, rt) (optional), 4 label eg. {$p_1$} (optional), 5 start and end positon of the feynman graph leg eg. {i,o}
\newcommand{\marrow}[5]{%
    \fmfcmd{style_def marrow#1
    expr p = drawarrow subpath (1/4, 3/4) of p shifted 6 #2 withpen pencircle scaled 0.4;
    label.#3(btex #4 etex, point 0.5 of p shifted 6 #2);
    enddef;}
    \fmf{marrow#1,tension=0}{#5}
}

% Introduce a time arrow with same usage properties
\newcommand{\tarrow}[5]{%
    \fmfcmd{style_def tarrow#1
    expr p = drawarrow subpath (1/6, 5/6) of p shifted 15 #2 withpen pencircle scaled 0.4;
    label.#3(btex #4 etex, point 0.5 of p shifted 15 #2);
    enddef;}
    \fmf{tarrow#1,tension=0}{#5}
}

% Introduce bundled and corved quark lines. Needs to be in seperate files, otherwise it breaks other Feynman graphs
% Use: \quark{unique label}{text label}{startpoint,endpoint of line e.g. 0,1}{Arrowhead possition e.g. 1.00, tension e.g. infinity}{shift the line e.g. 0,0}{position of the curve e.g. left} {shift the vertex e.g. 0,0}{shift the last point of a tangent e.g. 0,0}{from, to e.g. i3,v3}
\newcommand{\quark}[9]{
  \fmfcmd{input TEX;
    style_def quark#1 expr p =
    pair a, b, m, n;
    if "#6"="left":
      a = point 0 of p; b = point length(p) of p + (#7); m = point length(p) of p + (#8);
      path q; q = a{m-a}..tension ypart(#4)..{right}b;
      label.lft(btex #2 etex, point xpart(#3) of q shifted (#5))
    fi;
    if "#6"="right":
      a = point 0 of p + (#7); b = point length(p) of p; m = point 0 of p + (#8);
      path q; q = a{right}..tension ypart(#4)..{b-m}b;
      label.rt(btex #2 etex, point ypart(#3) of q shifted (#5))
    fi;
    cdraw subpath (#3) of q                          shifted (#5);
    cfill (tarrow (q,(xpart(#3)+ypart(#3))*0.46*xpart(#4))) shifted (#5);
    enddef;}
  \fmf{quark#1,tension=0}{#9}}

% Introduce braces for feynman graphs
\usepackage{scalerel}
% USE: \myleftbrace{Size in pt}{vspace in pt}{text space}
\newcommand{\mylbrace}[3]{\vspace{#2pt}\hspace{#3pt}\scaleleftright[\dimexpr5pt+#1\dimexpr0.06pt]{\lbrace}{\rule[\dimexpr2pt-#1\dimexpr0.5pt]{-4pt}{#1pt}}{.}}
\newcommand{\myrbrace}[3]{\vspace{#2pt}\scaleleftright[\dimexpr5pt+#1\dimexpr0.06pt]{.}{\rule[\dimexpr2pt-#1\dimexpr0.5pt]{-4pt}{#1pt}}{\rbrace}\hspace{#3pt}}

%%%%%%%%%%%%%%%%%%%%%%%%%%%%%%%%%%%%%%%%%%%%%%%%%%%%%%%%%%%%%%%%%%
% Glossary settings
%%%%%%%%%%%%%%%%%%%%%%%%%%%%%%%%%%%%%%%%%%%%%%%%%%%%%%%%%%%%%%%%%%
% Make glossary entries bold when first used
% \renewcommand*{\glstextformat}{\textbf}

% Reset all acronyms at the beginning of every new chapter
\preto\chapter{\glsresetall}

% Set acronym style to use first the long version then the short
\setabbreviationstyle[acronym]{long-short}

%%%%%%%%%%%%%%%%%%%%%%%%%%%%%%%%%%%%%%%%%%%%%%%%%%%%%%%%%%%%%%%%%%
% Deal with unessersary warnings
%%%%%%%%%%%%%%%%%%%%%%%%%%%%%%%%%%%%%%%%%%%%%%%%%%%%%%%%%%%%%%%%%%
% Suppress warning when there are multiple PDF-graphics on one page
\pdfsuppresswarningpagegroup=1

%%%%%%%%%%%%%%%%%%%%%%%%%%%%%%%%%%%%%%%%%%%%%%%%%%%%%%%%%%%%%%%%%%
% Stuff for Bob Ross Quote
%%%%%%%%%%%%%%%%%%%%%%%%%%%%%%%%%%%%%%%%%%%%%%%%%%%%%%%%%%%%%%%%%%
\newlength\boradest
