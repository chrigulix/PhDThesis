\chapter{Conclusion} \label{sec:conclusion}
Modern neutrino research relies on precise neutrino interaction models to accurately measure neutrino properties and to eventually find new physics. Presently, systematic model uncertainties are becoming a limiting factor for high precision neutrino interaction measurements like the \gls{dune}. Hence, it is paramount to improve said models and for this new high precision measurements are needed. This is especially the case for neutrino-argon interactions, as the next generation of large scale experiment, \ie \gls{dune}, will be using \gls{lartpc} technology as well, and there is only limited $\nu$-\ce{Ar} cross section information available so far. There are, however, issues with unfolding measured data into a cross section. In neutrino physics the unfolding process itself introduces model dependencies to the cross section result which increases systematic uncertainties. Another problem is posed by the inversion of the smearing matrix used to compensate the detector effects in the data. This leads to further increases in statistical and systematic uncertainties. In order to reduce model dependency and circumvent the matrix inversion, the differential and double-differential event distributions in this thesis are presented as forward-folded event rates. This means that, backgrounds and detector smearing is applied to the model predictions.

MicroBooNE's very first $\nu_\mu$ \gls{cc} inclusive measurement is described in chapter \ref{sec:FistCCInclusive}. It is based on my own analysis, using an early 2016 data set with \num{4.95e19} \gls{pot}, which was published as a public note \cite{MicroBooNECCInclPN}. The expected signal purity was found to be \SI{43}{\percent} and the efficiency at \SI{12}{\percent}. As expected for a surface detector like MicroBooNE, the cosmic-rays were the most prominent background of this measurement. Four models, all based on \gls{genie}, were compared to the forward-folded differential kinematic distributions of the muon, extracted from detector data. They were the baseline model ($M_A = \num{0.99}$), the baseline model with adjusted $M_A = \num{1.35}$, the \gls{mec} model, and a combined \gls{esf} with \gls{tem}. The latter was found to best match the measured $\cos{(\theta)}$ distribution, while $M_A = \num{1.35}$ most accurately represented the measured muon momentum distribution. Considering both distributions, \gls{esf}\&\gls{tem} was found to best match the measured signal overall. Moreover, I calculated the total $\nu_\mu$ \gls{cc} inclusive cross section of this measurement as $\sigma_{\nu_{\mu}} = ( \num{0.933} \pm 0.045 (\text{stat.}) \pm 0.146 (\text{syst.}) ) \times \SI{e-38}{\centi\metre\squared}$. This result is compatible with other measurements of this kind performed with MicroBooNE of $( \num{0.770} \pm 0.005 (\text{stat.}) \pm 0.113 (\text{syst.}) ) \times \SI{e-38}{\centi\metre\squared}$ \cite{CRTThomasPhD} and $( \num{0.693} \pm 0.010 (\text{stat.}) \pm 0.165 (\text{syst.}) ) \times \SI{e-38}{\centi\metre\squared}$ \cite{MicroBooNEFirstCCInclPublished}. My result is notably higher than the other two, although still in range of their respective uncertainties. In this data set the cosmic overlay sample did not yet exist and cosmic background simulation provided by \gls{corsika} was used. In my analysis, I discovered a notable mismatch between the angular distributions of measured cosmic-ray data and the simulation. Hence, the higher cross section value of my result is expected due to the increased drifted-in cosmic contamination left after my selection. This difference between cosmic data and simulation was then used to estimate the cosmic background systematic uncertainty.

In chapter \ref{sec:NewCCInclusive} MicroBooNE's most recent $\nu_\mu$ \gls{cc} inclusive measurement is described. This is part of a publication which is in progress. The used data sample originates from MicroBooNE's ``run 3'' with a total of \num{2.144e20} \gls{pot}. After the event selection, a purity of \SI{72}{\percent} and an efficiency of \SI{56}{\percent} are expected, according to simulations. Also here, the cosmic-rays were creating most of the background events. However, their number was greatly reduced by the use of advanced reconstruction and selection techniques, as well as by \gls{crt} vetos. This time, the result was presented as a forward-folded double-differential event rate as a function of muon momentum and $\cos{(\theta)}$. The model comparison was performed by utilising six samples produced by four different neutrino generators. The first three were generated by various \gls{genie} versions and tunes, while the other three originated from NEUT, NuWro, and GiBUU. Considering the calculated $\chi^2/\text{NDF}$ numbers it was found that the NEUT generator best fits this measurement. However, with best match featuring $\chi^2/\text{NDF} = \num{12.34}$, it can be stated that none of the models are an adequate fit for the data. In particular, the disagreement between the different models and the measured data is most pronounced in the high-momentum muon bins. This indicates that models, which are based on interaction phenomenology, do not suffice to make reliable $\nu_\mu$ \gls{cc} inclusive interaction predictions. As to why the model deviation is more pronounced in higher energy bins remains unclear, since the models themselves seem rather convoluted. I dare say that the field seems to be in dire need of more people and new impulses.

When comparing the two $\nu_\mu$ \gls{cc} inclusive interaction analyses, one finds the notable reconstruction and selection improvements the MicroBooNE collaboration was able to achieve. Most notable are the differences of the smearing matrices. While the ones from section \ref{sec:DetectorSmearing} are plagued by misreconstruction (not diagonal), the one of the double-differential distribution in section \ref{sec:NewDetectorSmearing} seems diagonal. Further, the quality differences appear in the signal to background ratios visible in the distributions, where the most recent study shows much reduced background levels. The background reduction is also represented by the differences in selection purity. Finally, the higher reconstruction quality allowed for less stringent selection regime in the latest measurement (see section \ref{sec:NewEventSelection}) compared to MicroBooNE's first measurement (see section \ref{sec:EventSelection}). This can be seen by comparing the selection efficiencies which is increased in the most recent analysis by a factor of six with respect to the first analysis. How the forward-folding influenced the reduction of systematic uncertainty is hard to quantify, for it would require a dedicated study comparing an unfolded and a forward-folded result stemming from the same selected data and \gls{mc} sets. In my view, forward-folding is clearly a superior data-model comparisons method, as opposed to unfolding and comparing cross sections. However, there are many human factors at play here, since forward-folding changes the workflow of established data analysts and modellers alike. Hence, I see an uncertain future for forward-folding.

Finally, the increased cosmic activity of a surface \gls{lartpc} not only impedes cross section related measurements, but also high sensitivity measurements. In the case of MicroBooNE, this is especially true for the $\nu_e$ \gls{lee} studies. The event signature of a $\nu_e$ is assumed as a single electron with relatively low energy. Said signature can also be mimiced by a cosmic gamma-ray interacting with the argon through Compton scattering. In the scope of my thesis, I created a simplified \gls{mc} simulation of cosmic gamma activity using \gls{cry} as a cosmic-ray generator. The study is described in detail in chapter \ref{sec:CosmicRayGammaBackground}. First, I verified \gls{cry}'s output by comparing it to the results of three cosmic gamma-ray experiments. Thereafter, the detector drift was simulated and several selection cuts were applied. According to my simulation, MicroBooNE experiences a cosmic gamma induced Compton electron rate which is 20 times higher than the possible $\nu_e$ \gls{lee} interaction rate. The energy limit for the electron was chosen to be $E > \SI{200}{\mega\electronvolt}$. If the energy cut value was reduced to $E > \SI{100}{\mega\electronvolt}$, the rate factor would increase to more than 80. This study later became one of the reasons for the development of the \gls{crt}. Anyhow, it is advisable to plan high sensitivity experiments with sufficient overburden. When reading the MicroBooNE proposal \cite{MicroBooNEProposal1,MicroBooNEProposal2} it becomes obvious that the collaboration heavily underestimated the cosmic background. I assume that the \gls{lartpc} technology was not properly understood and that they did not consider the relatively long drift time leading to cosmic pileup.
